\documentclass[a4paper, landscape]{jsarticle}
\usepackage{lscape}
\usepackage{tabularx}

\pagestyle{empty}
\title{2023年度春学期活動予定表}
\date{}
\begin{document}
\maketitle


\begin{table}[htbp]
  \centering
  \begin{tabular}{c|c|c|c}
    \hline
      活動名 & 活動内容 & 活動責任者 & 活動予定日 \\
    \hline \hline
      初心者向けプログラミング勉強会 & 
      AtCoderのA問題B問題中心にプログラミングの基本的な文法を学ぶ。 &
      未定 &
      未定 \\
    \hline
      アルゴリズム勉強会 &
      C問題D問題中心に基礎的なアルゴリズムを勉強する。 &
      未定 &
      未定 \\
    \hline
     鉄則本輪読会 &
     鉄則本を参加者で読み進める。 &
     小山修生 &
     未定 \\
    \hline
     蟻本輪読会 &
     蟻本を参加者で読み進める。 &
     佐々木優真 &
     未定 \\
    \hline \hline
     高度アルゴリズム・データ構造勉強会 &
     少し複雑なアルゴリズムやデータ構造を学ぶ。 &
     佐々木優真 &
     不定期 \\
     \hline
     ABC感想戦 &
     コンテスト後にコンテストの感想を話したり、解きなおしたりする。 &
     なし &
     ABC後(基本的に土曜夜)\\
     \hline
     バーチャルコンテスト &
     みんなでばちゃを走る &
     未定 &
     不定期 \\
     \hline
  \end{tabular}
\end{table}

\section{初心者向けプログラミング勉強会}
\noindent
対象:プログラミング初心者\\
\indent
プログラミングの文法を学び、簡単な問題を通してコードを書く練習をします。\\
ただプログラミングの理論を学ぶだけでなく、実際に手を動かして定着させられることが魅力です。\\
言語はc++またはPythonの好きな方を選んでいただいて結構です。

\section{アルゴリズム勉強会} \label{sec:algo}
\noindent
対象:基本的な文法は履修済みでアルゴリズムの勉強を始めたい人\\
\indent
AtCoderの問題でdifficulty$\sim$800程度で出題される題材を中心にアルゴリズムの勉強をします。\\
また、「計算量」という競技プログラミングにおいてもっとも重要な概念を勉強します。\\
プログラミングの基礎は勉強したけどアルゴリズムは全くわからないという方におすすめです。\\
例えば、効率的に迷路を解くアルゴリズム(BFS, DFS)や数列の区間和を高速に求める方法(累積和)等を学びます。

\section{鉄則本輪読会}
\noindent
対象:AtCoderのレートで水や青を目指したい人\\
\indent
『競技プログラミングの鉄則』(通称:鉄則本)を輪読します。\\
最初の方は\ref{sec:algo}アルゴリズム勉強会の内容とある程度被ると思うので、アルゴリズムが全くわからないという方でも大丈夫です。
こちらのほうがより深い内容までカバーしています。

\section{蟻本輪読会}
\noindent
対象:鉄則本では物足りない人、さらなる高みを目指したい人 \\
\indent
『プログラミングコンテストチャレンジブック』(通称:蟻本)を輪読します。\\
特にICPCで高い成績を目指したい人におすすめです。\\
蟻本に書いてある内容を理解するのはもちろんのこと、それに付随する内容やさらに高度な内容まで深堀していけたらと思っています。\\
まだあまり自信がない方やアルゴリズムを知らない方の参加も大歓迎です!
一緒に強くなりましょう。
\\

以上4つの活動は、難易度順になっているつもりです(複数に参加していただいてもokですし、途中からの参加や変更等も大丈夫です)。
\\

\section{高度アルゴリズム・データ構造勉強会}
基本的には不定期で、やりたいことができた際に開催します。\\
自分で勉強しようとして躓いてしまったものや、みんなで勉強してより理解を深めたいものがあれば、責任者佐々木(momoyuu)まで言ってください。\\
高度アルゴリズムと書いてありますが、基本的なアルゴリズムをより確実に身に着けたいといった内容も大歓迎です。

\section{ABC感想戦}
ABCの後(場合によってはARCやAGC、codeforcesの後)にDiscordの通話で感想を語り合ったり、解けなかった問題を強い人に教えてもらったりします。\\
時間が遅かったり、理工学部生はレポートに追われていたりするので参加は任意です。\\
自分ができなかった問題を復習するいい機会だと思うので是非参加してみてください。

\section{バーチャルコンテスト}
誰でも勝手にバチャを開いてくれて大丈夫です。\\
需要があれば定期的に開催するかもしれません。\\
また、ある程度大きいバチャや対面バチャもやりたいなあと思ったりしています。\\

\end{document}