\documentclass{ltjsarticle}
\usepackage{hyperref, enumitem, graphicx}

\title{Ants規約}
\date{\today}
\author{Ants発足委員会}
\begin{document}
  \maketitle
  \thispagestyle{empty}
  \addtocounter{page}{-1}
  \clearpage
  \setcounter{tocdepth}{3}
  \tableofcontents
  \clearpage
  \newcounter{cnt}
  \setcounter{cnt}{0}
  \newcommand{\jor}{
    \refstepcounter{cnt}
    \textbf{第\the\value{cnt}条: }
  }
  \section{総則}
    \subsection{名称}
      \jor
      この団体の名称は「Ants」とする\footnote{以下本団体と呼称する.}.
    \subsection{事務所}
      \jor
      本団体は事務所を,団体代表自宅に置く.
    \subsection{用語}
      \jor
      本団体は4月1日から9月30日を上半期,10月1日から翌年3月31日を下半期とする.
      上半期,下半期を1クールとする.9月30日と3月31日をクール末とする.
      また4月1日から翌年3月31日を通年と呼称とする.
    \subsection{目的}
      \jor
      本団体は競技プログラミングを中心とした数学,情報学とその関連分野を半学半教の精神に則り学習及び研究を行うことを,また,同活動を通して会員間の親睦を図ることを目的とする.
  \section{会長,会員,組織および役員}
    \subsection{会長}
      \jor
      本団体を統括する者として,会長を置き,会長には慶應義塾大学の教授,准教授もしくはこれに準ずる専任の教員を充てる.
    \subsection{会員}
      \subsubsection{身分とそれを獲得する要件}\label{certificate}
        \jor
        本団体は正会員,準会員,特別会員と仮会員の4種の会員の身分を持つ.
        \begin{enumerate}
          \item 正会員は期限付きの身分であり,正会員となった日から初めてクール末を迎えるまでに,別途定める継続の手続き\footnote{「会則(継続の手続きについて)」を参照.}を行わない場合はその身分を失う.
          \item 正会員は次の要件を全て満足しなければならない.
                \begin{itemize}
                  \item 慶應義塾大学の学部学生であること.
                  \item 在籍を希望する当該期間分の会費を納めること.
                  \item 別途定める書類\footnote{「会則(入退会について)」を参照.}が受理されること.
                \end{itemize}
          \item 準会員は期限付きの身分であり,準会員となった日から初めてクール末を迎えるまでに,別途定める継続の手続き\footnote{「会則(継続の手続きについて)」を参照.}を行わない場合はその身分を失う.
          \item 準会員は次の要件をすべて満足しなければならない.
                \begin{itemize}
                  \item 慶應義塾大学の大学院生もしくは別科・日本語教育研修課程学生または特別短期留学生であること.
                  \item 別途定める書類\footnote{「会則(入退会について)」を参照.}が受理されること.
                \end{itemize}
          \item 特別会員は\ref{condition}にて規定する場合を除き,失われることはない.
          \item 特別会員は次の全ての要件を全て満足しなければならない.
                \begin{itemize}
                  \item 塾員であること.
                  \item 過去に1クール以上正会員もしくは準会員であったこと.ただし,学生代表が認めた場合はその限りではない.
                  \item 別途で定める書類\footnote{「会則(入退会について)」を参照.}が受理されること.
                \end{itemize}
          \item 仮会員は期限付きの身分であり,仮会員となった日から起算して4ヶ月目の1日にてその身分を失う.
          \item 仮会員は次の全ての要件を満足しなければならない.
                \begin{itemize}
                  \item 慶應義塾大学の学部学生であること.
                  \item 過去に正会員もしくは仮会員であった期間が1クール以上ないこと.
                  \item 別途で定める書類\footnote{「会則(入退会について)」を参照.}が受理されること.
                \end{itemize}
        \end{enumerate}
      \subsubsection{会員資格を失う場合}\label{condition}
        \jor
        \ref{certificate}にて定めた会員たる要件を満たさなくなった場合に加えて,各会員は次に該当する場合に会員資格を失う.
        \begin{enumerate}
          \item 会員が死亡した場合.
          \item 別途定める手段\footnote{「会則(入退会について)」を参照.}にて,会員からの通知があった場合.
          \item 2回連続で総会を欠席し,共に委任状の提出がない場合.
          \item 懲罰会\footnote{\ref{pconf}を参照.}で退会の処分を受けた場合.
        \end{enumerate}
    \subsection{役職}
      \subsubsection{役職とその定員}
        \jor
        本団体の正会員は以下の役職を持つことができ,それぞれの役職は学生代表,学生副代表,会計局長,広報局長,渉外局長であり,定員は全て1名である.
        また,役職を持つ正会員を役員と呼称し,役職を持たないものを一般正会員と呼び区別する.
        それぞれの役員の職務,任期及び選出方法は\ref{jobs}にて定める.
      \subsubsection{役員と一般正会員の兼任の禁止}
        \jor
        正会員が同時に役員かつ一般正会員たることはできない.
    \subsection{監査部長}
      \jor
      一般正会員のうちから1名監査部長を置く.監査部長の職務,任期及び選出方法は\ref{jjobs}にて定める.
    \subsection{組織}
      \jor
      本団体は以下に示す組織を内部に持つ.
      \begin{enumerate}
        \item 本団体には次の組織を設ける. 執行部,監査部.
        \item 前項で定めた執行部は,\ref{sconf}にて定めた総会を経て可決された予算の執行を行う.
        \item 第1項で定めた執行部の下に,次の組織を設ける. 会計局,広報局,渉外局,総務局.
        \item 前項で定めた会計局,広報局,渉外局,総務局は,局長の指名をもって,正会員による局員を構成することができる.この際,\ref{cab}で定める役員会の可決を必要とする.
        \item 第1項で定めた監査部は,執行部によって執行された予算の妥当性を総会にて報告する.
        \item 前項で定めた監査部の下に,次の組織を設ける. 会計監査局,懲罰局.
      \end{enumerate}
    \subsection{役員}
      \subsubsection{役員の職務}\label{jobs}
        \jor
        役員の職務は下表の通りである.役員による役職の兼任は妨げられない.
        \begin{enumerate}
            \item 学生代表は,本団体の学生責任者である. 
            \item 学生副代表は,学生代表が一時的に職務を行うことができない場合に学生代表に代わり職務を行う.
            \item 会計局長は,予算案を編成し,総会に提出する.また,予算の執行に関して全責任を負う.
            \item 広報局長は,本団体の周知を目指す活動の責任者である.
            \item 渉外局長は,本団体の対外活動に対して企画立案を行い,またその責任者である.
            \item 総務局長は,本団体の外部との交流を伴わない活動に対して企画立案を行い,またその責任者である.
        \end{enumerate}
      \subsubsection{役員の任期}
        \jor
        学生副代表を除く全役員の任期は通年である.ただし通年の途中に選出された役員の任期は,前任者の予定されていた任期と同一である.
      \subsubsection{役員の選出方法}
        \jor
        役員は立候補に基づく総会での投票で選出される.学生副代表の任期は新たに学生代表が選出されるまでである.
        役員候補は立候補時に,任期満了時に会員でなければならない.
      \subsubsection{役員の罷免}
        \jor
        役員は次の場合に罷免される.
        \begin{enumerate}
          \item 心身の故障,またはその他の事由により職務の遂行が,相当長期にわたり困難である認められた場合
          \item 通常総会,もしくは臨時総会にて罷免動議が提出され,可決された場合.
          \item 正会員でなくなった場合.
        \end{enumerate}
        次期役員が選出されるまでの間は\ref{jobs}で定めた順位の高い役員のうち,罷免されていない役員が臨時的に兼任し,臨時役員と呼ぶ.
        また,全役員が罷免された場合は,次期役員が選出されるまでの間,本団体は通常総会もしくは臨時総会以外の活動を停止する.
    \subsection{監査部長}\label{jjobs}
      \subsubsection{監査部長の職務}
        \jor
        監査部長は会計監査局長と懲罰局長を兼任する.
        会計監査局長は予算の監査を予算案の総会への提出時に行い,修正の必要があれば総会時に全ての正会員にわかるように通知しなければならない.
        また,決算に関する監査をクール末に行い,次クールの総会にてその結果を総会に提出しなければならない.
        懲罰局長は,必要に応じて懲罰会を開催しなければならない.
      \subsubsection{監査部長の任期}
        \jor
        監査部長の任期は通年である.
      \subsubsection{監査部長の選出方法}
        \jor
        監査部長は役員会が任命し,通常総会もしくは臨時総会で承認を受けなければならない.
      \subsubsection{監査部長の罷免}
        \jor
        監査部長が罷免される場合は以下の通りである.
        \begin{enumerate}
          \item 心身の故障,またはその他の事由により職務の遂行が,相当長期にわたり困難である認められた場合.
          \item 正会員でなくなった場合.
        \end{enumerate}
        監査部長が罷免された場合は,次期監査部長が選出されるまでの間,本団体は通常総会もしくは臨時総会以外の活動を停止する.
  \section{予算}
    \subsection{予算編成}
      \jor
      予算案は通年のものを執行部会計局が作成し,監査部長と総会に提出されなければならない.
    \subsection{会計監査}
      \jor
      監査部長は新年度予算案を会計局長から受け取り,その内容が適切であるかを現年度と比較しながら検討し,適切か否かを総会時に全ての正会員に通知しなければならない.
      また,クール末には予算の執行状況を確認し,それに関する報告を次クールの総会で行わなければならない.
  \section{会議}
    \jor
    会議の種類によらず,学生副代表は議事録を作成し,それを全会員に公開しなければならない.
    議事録の作成及び公開方法について別途定める\footnote{「会則(議事録について)」を参照.}.
    \subsection{通常総会}\label{sconf}
      \subsubsection{議題}
        \jor
        通常総会では提出された予算案と,会員から提出された議題を審議するものとする.
      \subsubsection{開催時期}
        \jor
        本団体は上半期が始まってから1ヶ月以内に通常総会を開催しなければならない.
        議長は監査部長とする.
        議題の提出方法は別途定める\footnote{「会則(議題の提出について)」を参照.}.
      \subsubsection{参加するべき会員}
        \jor
        通常総会に参加するべき会員とその権利については下に定める.
        \begin{enumerate}
          \item 正会員は議決権及び,議題の提出権を持ち,通常総会に参加しなければならない.
          \item 前項にも関わらず,相応の理由をもって通常総会を欠席する場合は細則にて定める手続き\footnote{「会則(委任状について)」を参照.}を経て議決権を議長に委任しなければならない.  
          \item 準会員は通常総会に参加することができる.準会員は議決権はもたないが,議題を提出することができる.
          \item 特別会員は通常総会に参加することができる.ただし,特別会員は議決権や議題の提出権をもたない.
          \item 仮会員の通常総会への参加はこれを認めない.
        \end{enumerate}
      \subsubsection{開催方法}
        \jor
        原則として対面で行い,開催地は学生副代表が決定する.しかし,全正会員の5分の1以上が対面で参加不可能であり,オンラインでは参加可能な場合はオンラインの開催を認める.
        オンラインでの開催方法は別途定める\footnote{「会則(会議の開催方法について)」を参照.}.
      \subsubsection{議決方法}
        \jor
        正会員の人数の過半数の賛成が得られた場合,議案は可決されたとする.
        委任された議決権は議長の裁量で投票できる.
    \subsection{臨時総会}\label{cab}
      \subsubsection{議題}
        臨時総会では補正予算案と会員から提出された議題を審議するものとする.
      \subsubsection{開催時期}
        \jor
        役員のうち,少なくとも1人の要求があった場合または,一般正会員の10分の1以上の要求があった場合に臨時総会を開催する.
        議題の提出方法は別途定める\footnote{「会則(議題の提出について)」を参照.}.
        学生代表は議事録を作成し,全会員に周知しなければならない.
      \subsubsection{参加するべき会員}
        \jor
        参加するべき会員とその権利は通常総会に準ずる.
      \subsubsection{開催方法}
        \jor
        開催方法は通常総会に準ずる.
      \subsubsection{議決方法}
        \jor
        補正予算案は特別多数\footnote{出席した議決権をもつ会員の3分の2以上を指す.}の賛成を要する.その他の議案については通常総会に準ずる.
    \subsection{役員会}
      \subsubsection{議題}
        \jor
        役員会では本団体の1ヶ月の運営方針を話し合うものとする.
      \subsubsection{開催時期}
        \jor
        毎月15日までに開催しなければならない.
        議長は学生代表とする.
      \subsubsection{参加するべき会員}
        \jor
        全役員は必ず出席しなければならない.また,各役員につき1名,一般正会員をオブザーバとして参加させることができる.
      \subsubsection{開催方法}
        \jor
        開催方法は通常総会に準ずる.
      \subsubsection{議決方法}
        \jor
        原則として全会の一致した意見が得られれば,議題は成立とする.
        しかし,同一の議案に対して2回議決を行い,なおも全会の一致を得られない場合は5人の役員の賛成によって成立とする.
    \subsection{懲罰会}\label{pconf}
      \subsubsection{議題}
        \jor
        懲罰会では開催するに至った事実の確認と処分の決定を行う.処分内容は\ref{punishment}で定める.
      \subsubsection{開催時期}
        \jor
        次のいずれかに該当する場合,懲罰局長は懲罰会を開催しなければならない.
        \begin{enumerate}
          \item 20歳未満の会員が活動中に飲酒行為を行った場合.
          \item 会員が他の20歳未満の会員に飲酒を勧めた場合.
          \item 他の会員への飲酒の強要を行った場合.
          \item 性及び性的嗜好,人種,国籍,出身地,信教,その他の思想信条についての差別的な言動を行った場合.
          \item 不正会計が行われた場合.
          \item 懲罰動議が監査部長に提出された場合.
        \end{enumerate}
        議長は監査部長とする.
      \subsubsection{参加するべき会員}
        \jor
        懲罰会を開催するに至った事由に関わる会員に加えて,全役員と,抽選で選ばれた一般正会員は参加しなければならない.
        ただし,全役員数以上の一般正会員を抽選で選ばなければならない.
      \subsubsection{開催方法}
        \jor
        開催方法は通常総会に準ずる.
      \subsubsection{処分内容}\label{punishment}
        \jor
        懲罰会は以下のうちのから1つを議決しなければならない.
        \begin{enumerate}
          \item 証拠不十分.
          \item 戒告.
          \item 役員の解任(役員が懲罰対象の場合のみ).
          \item 有期の活動停止.
          \item 無期の活動停止.
          \item 一定期間後の再入会を認める強制退会.
          \item 再入会を認めない強制退会.
        \end{enumerate}
      \subsection{議決方法}
        \jor
        出席した会員の特別多数をもって処分内容を決定する.議決に至らない場合は処分はくだされない.
  \section{財源}
    \subsection{通常会費}
      \jor
      本団体の正会員は,通常会費を納めなければならない.
      通常会費は総会によって定められた予算案に従う.
    \subsection{臨時会費}
      \jor
      本団体の正会員は,臨時会費を納めなければならない.
      ただし,臨時会費は会計局が臨時総会に提出した補正予算案が可決されたときに限り徴収を認める.
    \subsection{寄付金}
      \jor
      本団体は常に寄付金を受け付ける.
      納入された通常会費は理由の如何に関わらず返還しない.
    \subsection{その他の収入}
      \jor
      本団体の活動を通じて得た報酬や利益は,すべて本団体の収入とする.
  \section{危機管理}
    \subsection{事前準備}
      \subsubsection{危機管理責任者}
        \jor
        本団体が会員同士が対面して活動を行う場合,活動ごとに危機管理責任者を定める.
        危機管理責任者は定められた書類に,活動場所から最も近い医療機関等の場所と参加者の名前を記入し,監査部に提出しなければならない\footnote{「会則(活動申請)」を参照.}.
        危機管理責任者は原則として総務局長がつとめるが,相応の理由がある場合は,活動に参加する会員に代行させなければならない.
      \subsubsection{保険}
        \jor
        本団体が指定する保険に加入していない会員は,対面での活動には参加できない.
      \subsubsection{課外活動}
        本団体の名のもとで会員が活動を行う場合は,大学への所定の手続きを必要とし,危機管理責任者がその責任を負う.
        ただし会員同士が対面せず活動を行う場合においては必ずしもこれに依らない.
    \subsection{活動中}
      \subsubsection{飲酒を伴わない活動の場合}
        \jor
        活動中,危機管理責任者は参加した会員の健康状態を定期的に確認しなければならない.
        また,活動中に事故等が発生した場合には,危機管理責任者は活動に参加した会員と協力して必要な措置を講じるとともに,直ちに会長及び大学に至急連絡をしなければならない.
      \subsubsection{飲酒を伴う活動の場合}
        \jor
        危機管理責任者は,事前に作成した色付きのリストバンドを,各会員が20歳以上であるかを区別して,全員に着用させなければならない.
        また,活動に参加する会員の中から適当な人数を危機管理副責任者として任命し,協力して会員の健康状態を及び20歳未満会員の飲酒の有無を確認しなければならない.
        危機管理責任者と危機管理副責任者は飲酒をしてはならない.
        確認の頻度は30分に1回か,それよりも頻繁に行わなければならない.
        活動中に事故等が発生した場合には,危機管理責任者は活動に参加した会員と協力して必要な措置を講じるとともに,直ちに会長及び大学に至急連絡をしなければならない.また,懲罰委員会を必ず開催しなければならない.
  \section{細則}
    \subsection{会則}
      \subsubsection{制定方法}
        \jor
        会則は\ref{sconf}で定めた通常総会もしくは臨時総会にて可決されなければならない.
    \subsection{本規約}
      \subsubsection{改正方法}
        \jor
        本規約は会則の制定と同様の手続きをもって改正される.
  \section*{附則}
    本規約は2023年4月4日に公布し,本団体が慶應義塾大学学部学則に基づく公認を受けた日から施行する.
\end{document}